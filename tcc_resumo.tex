\documentclass{article}
\usepackage[portuguese]{babel}
\usepackage[utf8x]{inputenc}
\usepackage{xcolor}

\title{Desenvolvimento de base de dados para treinamento de redes neurais de reconhecimento de voz
através da geração de aúdios com resposta ao impulso simuladas por técnicas de data augmentation [RESUMO]}
\author{Bruno Machado Afonso}

\begin{document}
\maketitle

\section{Introdução}

Com o avanço das tecnologias de automação residencial, assistentes pessoais nos
smartphones e comunicação online, o estudo de técnicas de processamento de áudio (nesse 
caso específico, voz), torna-se cada vez mais relevante para a sociedade.
Junto a isso, houve grandes avanços no âmbito do aprendizado de máquina, fornecendo
alternativas para os métodos tradicionais de processamento de áudio.

Modelos de arquitetura necessitam de um grande volume de dados para que sejam 
treinados e aprimorados, e um dos mais recentes desafios é de capturar essa extensa 
quantidade de gravações de áudio, pois é uma tarefa alto custo tanto financeiro e 
temporal, necessitando de equipamento especializado e diversos locais com sons de fundo
e pessoas para amostras de voz.

A proposta dessa tese é de gerar uma base de dados, contendo amostras de voz reverberadas
em diversos ambientes, para treinamento de redes neurais, 
partindo de amostras de voz e resposta ao impulso de salas já existentes.
A tese propõe em utilizar técnicas de data augmentation para gerar um grande volume
de vozes reverberadas à partir de um pequeno conjunto de amostras.


\section{Resposta ao Impulso de uma sala (RIR) e técnicas de Data Augmentation}

\textcolor{red}{\LARGE{[Falar sobre RIR]}} \newline



\textcolor{red}{\LARGE{[Falar sobre artigo de DA no RIR]}} \newline
\textcolor{red}{\LARGE{[Falar sobre artigo de ruidos em voz]}} \newline


\cite{RIR_Data_Aug}
\cite{Speech_Rec}

\bibliography{tcc_resumo_bib} 
\bibliographystyle{ieeetr}

\end{document}